\section{接口,lambda表达式,内部类}
\subsection{接口}
\subsubsection{接口的概念}

在 Java 程序设计语言中,接口不是类,而是对希望符合这个接口的类的一组需求。

下面我们看一个 \texttt{Comparable} 接口\footnote{在旧版 Java 中,使用的是 Object 而不是泛型类型。虽然泛型类型要在之后说明,但现在已经很少用 Object 进行强制类型转换,泛型已经成为绝大多数程序员的首选。}。
\begin{Java}
public interface Comparable<T> {
    int compareTo(T other);
}
\end{Java}
这说明,任何实现该接口的类都必须包含 \texttt{compareTo} 方法,这个方法有一个泛型参数 \texttt{other},并返回整数。

接口中的所有方法都是 \texttt{public}。因此在接口中声明方法时,不必提供关键字 \texttt{public}。

接口中不应该提供实例字段和实例方法(新版 Java 已经可以提供简单的实例方法)。这个任务应该交给实现接口的类来完成。

为了让类实现一个接口,通常需要完成以下几步操作。
\begin{enumerate}
    \item 将类声明为实现给定的接口。
\begin{Java}
class Employee implements Comparable<Employee>
\end{Java}
    \item 对接口中的所有方法提供定义。
\begin{Java}
class Employee implements Comparable<Employee> {
    public int compareTo(Employee other) {
        return Double.compare(this.salary, other.salary);
    }
}
\end{Java}
\end{enumerate}

\fbox{
    \parbox{0.87\textwidth}{
        \begin{warning}
            在接口申明中可以省略 \texttt{public}。但是在实现接口时,必须把方法声明为 \texttt{public},否则编译器会报错。
        \end{warning}
    }
}

既然接口只有一个方法,那么我们为什么不直接实现该方法,而要多一步实现接口的操作呢? 这主要有两个原因:
\begin{itemize}
    \item 一是在参数类型中我们可以书写接口类型,这样就能确保实参都实现了该接口。方便后续操作。
    \item 二是 Java 是一种工程化语言,往往具有复杂的继承关系,实现接口方便项目管理。
\end{itemize}

\subsubsection{接口的属性}

接口不是类,不能使用 \texttt{new} 运算符实例化一个接口。但能却能声明接口变量,接口变量必须引用实现了这个接口的类对象:
\begin{Java}
Comparable x = new Employee("Pionpill");
\end{Java}

接口在许多方面表现的都与类一样,比如接口变量可以使用 \texttt{instance} 检查,接口可以继承接口。接口中的成员有以下特点:
\begin{itemize}
    \item 成员方法: 声明时必须为 \texttt{public},可以省略。实现时必须为 \texttt{public},但不可以省略。
    \item 成员属性: 接口中不能包含实例字段,但是可以包含常量。接口中的字段默认为 \texttt{public static final}。
\end{itemize}

既然现代的接口已经可以实现某些简单的方法,那么抽象类与接口的差别在哪呢。设计接口的最主要原因是 Java 只允许类的单继承,而类在实现接口时可以多继承:
\begin{Java}
class Employee implements Cloneable, Comparable
\end{Java}

\subsubsection{静态和私有方法}

在 Java8 中,允许在接口中添加静态方法。理论上讲,这没有任何语法错误,但是这有悖于接口的初衷。目前为止,通常的做法是将静态方法放在伴随类中。在标准库中,经常会有成对出现的接口和实用工具类,如 \texttt{Collection/Collections}。

随着 Java 的更新,在 Java11 标准库中已经将很多静态方法放在接口中,这样一来,这就没有必要再为实用工具方法另外提供一个伴随类。

在 Java9 中,接口中的方法是可以为 \texttt{private} 的。既可以修饰静态方法也可以是实例方法。但作用十分有限。

\subsubsection{默认方法}

可以为接口方法提供一个默认实现。必须用 \texttt{default} 修饰符标记这样一个方法。

\begin{Java}
public interface Comparable<T> {
    default int compareTo(T other) {
        return 0
    }
}
\end{Java}

当然这并没有太大的用处,毕竟每一个实现接口的类都会覆盖里面的方法。

不过某些情况下,这会有些用:
\begin{itemize}
    \item 接口演化问题: 如果随着接口更新,加入了新的方法,对应的实现类就会报错,这时候设置一个默认方法就可以避免这种错误。
    \item 默认方法可以调用其他方法,这可以简化一些操作。
\end{itemize}

既然接口可以多继承,那必然会导致同名方法冲突,幸运的是,Java 解决这类冲突的规则十分简单:
\begin{itemize}
    \item 超类优先。如果超类与接口具有同名方法,那么采取超类中实现的具体方法。
    \item 接口冲突。如果两个接口提供了相同的默认方法,那么必须覆盖这个方法来解决冲突。
\end{itemize}

多继承带来的冲突远不止这些,但 Java 的解决思路大致如上,如果具体的实现类与接口冲突,则才需类中的方法,接口有没有提供默认方法并灭有什么区别。这正是 ``类优先'' 原则。

\subsubsection{回调}

回调简单来讲是指某一件事发生后进行调用。