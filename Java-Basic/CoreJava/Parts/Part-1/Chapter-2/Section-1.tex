\chapter{面向对象}
\section{对象与类}
\subsection{包}
\subsubsection{包名}

使用包的主要原因是确保类名的唯一性。一般,为了确保包名的绝对唯一性,要用一个因特尔域名以逆序的形式作为包名,然后对于不同的工程使用不同的子包。

从编译器的角度来看,嵌套的包之间没有任何关系。例如,\texttt{java.util} 与 \texttt{java.util.jar} 包毫无关系。每一个包都是独立的类集合。

\subsubsection{类的导入}

一个类可以使用所属包中的所有类,以及其他包中的公共类。

可以使用两种方式访问另一个包中的公共类。第一种方式是使用完全限定名;就是包名后面跟着类名:

\begin{Java}
java.time.LocalDate today = java.time.LocalDate.now();
\end{Java}

这种方式很繁琐,更常见的是使用 \texttt{import} 语句引入:

\begin{Java}
import java.time.*;
\end{Java}

上面这种使用通配符导入所有类对代码的规模没有任何负面影响。不过如果能精确到类名,可以增加代码可读性。此外,* 可以导入目标包下的所有类,但是不能导入文件下的包。

在极个别情况下,仍有可能出现类名冲突,这时候就只能使用完全限定名了:

\begin{Java}
import java.util.*;
import java.sql.*;
var deadline = new java.util.Date();
var today = new java.sql.Date(...);
\end{Java}

此时也体现了 \texttt{var} 声明语句的好处:可以自动判断数据类型,省去了大段代码。

\subsubsection{静态导入}

有一种 \texttt{import} 语句允许导入静态方法和静态字段,而不只是类。

比如,在源文件顶部,添加一条指令:
\begin{Java}
import static java.lang.System.*;
out.println("Goodbye");
\end{Java}

另外,还可以导入特定的方法或字段:
\begin{Java}
import static java.lang.System.out;
\end{Java}

\subsubsection{类路径}

类路径是所有包含类文件的路径的集合。
在 UNIX 环境中,类路径用 : 分隔,在 Windows 环境中,则以 ; 分隔。

设置类路径可以使用 -classpath 选项,也可以通过设置 CLASSPATH 环境变量来指定。

\begin{Java}
java -classpath c:\classdir;,;c:\archives\archive.jar MyProg
set CLASSPATH = c:\classdir;.;c:\archives\archive.jar
\end{Java}

直到推出 shell 为止,类路径设置均有效。

\subsection{JAR 文件}

Java 归档文件(JAR) 文件既可以包含类文件,也可以包含诸如图像和声音等其他类型的文件。此外,JAR 文件是压缩的,它使用了我们熟悉的 ZIP 压缩格式。

\subsubsection{创建 JAR 文件}

通常,jar 命令的格式如下:
\begin{Java}
jar options file1 file2 ...
\end{Java}

其中 options 包含的可选项如下:

\begin{table}[H]
    \centering
    \caption{jar 程序选项}
    \label{table:jar 程序选项}
    \setlength{\tabcolsep}{10mm}
    \begin{tabular}{c|ccc}
        \toprule
        \textbf{选项} & \textbf{说明} \\
        \midrule
        c & 创建一个新的或空的存档文件并加入文件。 \\
        C & 临时改变目录。 \\
        e & 在清单文件中创建一个入口点。 \\
        f & 指定 JAR 文件名作为第二个命令行参数。 \\
        i & 建立索引文件 \\
        m & 将一个清单文件添加到 JAR 文件中 \\
        M & 不为条目创建清单 \\
        t & 显示内容表 \\
        u & 更新一个已有的 JAR 文件 \\
        v & 生成详细的输出结果 \\
        x & 解压文件 \\
        0 & 存储但不进行 ZIP 压缩 \\
        \bottomrule
    \end{tabular}
\end{table}

\subsubsection{清单文件}

清单文件用于描述归档文件的特殊特性。清单文件被命名为 MANIFEST.MF,它位于 JAR 文件的一个特殊的 META-INF 子目录中。

复杂的清单文件可能包含更多条目。这些清单条目被分成多个节。第一节被称为主节。他作用于整个 JAR 文件。随后的条目用来指定命名实体的属性,如单个文件,包或URL。它们必须以一个 Name 条目开始。节于节之间用空行分开。例如:

\begin{Java}
Manifest-Version: 1.0
lines describing this archive

Name:Woozle.class
lines describing this class
Name:com.mycompany.mypkg/
lines describing this package

\end{Java}

要想编辑清单文件,需要将希望添加到清单文件中的行放到文本文件中,然后运行:
\begin{Java}
jar cfm jarFileName mainfestFileName...
\end{Java}

要想更新一个已有的 JAR 文件的清单,则需要将添加的部分放置到一个文本文件中,然后执行以下命令:

\begin{Java}
jar ufm MyArchive.jar manifest-additions.mf
\end{Java}

此外,清单文件的最后一行必须是一个换行符,否则将无法被正确读取。

\subsubsection{可执行 JAR 文件}

可以使用 jar 命令中的 e 选项指定程序的入口点:

\begin{Java}
jar cvfe MyProgram.jar com.mycompany.mypkg.MainAppClass files to add
\end{Java}

或者在清单文件中指定程序的主类:
\begin{Java}
Main-Class: com.mycompany.mypkg.MainAppClass
\end{Java}

无论使用哪一种方法,用户可以简单地通过下面的命令来启动程序:
\begin{Java}
java -jar MyProgram.jar
\end{Java}

在 Mac OS 平台中,可以直接通过双击 jar 文件执行程序,但在 Windows 平台中,Java 运行时安装程序将为 ".jar" 扩展名创建一个文件关联,会用 javaw -jar 命令启动文件。当然也可以通过第三方包安装工具将 jar 文件转换为 exe 文件执行。

\subsection{文档注释}

JDK 包含一个很有用的工具,叫做 javadoc,它可以由源文件生成一个 HTML 文档。这些文档的内容源于我们在程序中写的注释。

由于最终生成的是 HTML 文件,因此我们在注解中可以加入 html 标签。除了普通的文本,我们还可以写入自由格式文本,标记以 @ 开始,如 @author。下面将重点介绍这些自由格式文本。

\subsubsection{通用注释}

通用注释含有的标记如下:
\begin{itemize}
    \item \texttt{@since text}
    
    始于条目,文本可一世引入这个特性的版本的任何描述。

    \item \texttt{@author name}
    
    作者条目,每个作者条目对应一个人名,可以有多个作者条目

    \item \texttt{@version text}
    
    版本条目
\end{itemize}

\subsubsection{方法注释}

方法中特有的标记如下:
\begin{itemize}
    \item \texttt{@param variable description}
    
    用于描述参数。
    \item \texttt{@return description}
    
    用于描述返回值。

    \item \texttt{@throws class description}
    
    用于描述可能抛出的异常。
\end{itemize}

\subsubsection{包注释}

要产生包注释,就需要在每一个包目录中添加一个单独的文件。可以有如下两个选择:
\begin{itemize}
    \item 提供一个名为 package-info.java 的文件,在里面使用 /** */ 文档注释,后面是一个 package 语句。它不能包含更多的代码或注释。
    \item 提供一个名为 package.html 的文件。会抽取 body 标签内的所有文本。
\end{itemize}

\subsubsection{注释抽取}

如果需要要获取 javadoc 文件,则只需执行对应的 javadoc 指令即可。

\newpage